\part{Avant de déboger}%
\label{part:before}

\paragraph{} Avant de commencer à s'aventurer plus loin dans les méandres du
débogage, il faut d'abord vérifier et faire quelques petites choses sur la base
de code.

\section{Bonnes pratiques}

\subsection{Variables}

\paragraph{} Même si il peut paraître plutôt chiant de respecter certaines
conventions quand on code, la vraie utilité des bonnes pratiques prend son
apothéose lorsque quelque chose ne marche pas. Par exemple, nommer ces
variables avec une seule lettre (à l'exception de l'éternel \texttt{i} comme
variable d'itération) peut devenir la source de vos problème si vous devez
revenir sur un bout de code écrit il y a un peu de temps.

\paragraph{} De manière générale, le code doit être fait de sorte que l'on
puisse le lire naturellement:

\begin{figure}[H]
	\centering
	\begin{minted}{python}
a = 'acgatagc'
b = len(a) - 2
d = ""
for e in range(0, f, 3):
  g = a[e:e+3]
  h = i.get(g.upper(), 'X')
  d = d + h
	\end{minted}
	\caption{Un mauvais exemple}
\end{figure}

\begin{figure}[H]
	\centering
	\begin{minted}{python}
dna              = 'acgatagc'
last_codon_start = len(dna) - 2
protein          = ""
for codon_start in range(0, last_codon_start, 3):
  codon      = dna[codon_start:codon_start+3]
  amino_acid = genetic_code.get(codon.upper(), 'X')
  protein    = protein + amino_acid
	\end{minted}
	\caption{Un bien meilleur exemple}
\end{figure}

\paragraph{} Dans les exemples juste avant, on voit bien que dans les deux cas,
on ne comprends pas trop ce qui se passe. Mais au moins, dans le deuxième cas,
on sait de quoi le code parle.

\paragraph{} Ça n'a pas l'air de grand chose comme ça, mais il ne faut surtout
pas avoir peur de mettre de long nom de variable dans votre code (mais quand
même avec modération), ça vous sauvera sûrement plusieurs heures de souffrance
si les choses tournent mal.

\paragraph{Conclusion:} Ne suivez pas aveuglement les noms de variables de ces
algorithmes qui sont tous étonnement composé d'une seule lettre et qui vont à
coup sûr vous donner une gueule de bois le lendemain.

\subsection{Unicité}

\paragraph{} Il faut qu'on parle. D'unicité. Il ne s'agit pas du truc qui
permet aux héros de battre tous ensemble l'ennemi super méchant dans les
blockbusters américain. Bien que ça soit moins intéressant à regarder,
l'unicité a des propriétés bien pratique. Il s'agit, dans le cas de la
programmation, du fait qu'une fonction ne doit faire qu'une et une seule chose.

\paragraph{} Appliquer cette théorie dans votre base de code vous permettant
non seulement de rendre votre code beaucoup plus lisible, mais aussi vos
donnera de l'inspiration pour vos nom de fonction. Si jamais vous n'arrivez pas
à trouver un nom de fonction dans la seconde, c'est sûrement qu'il y a un
problème.

\begin{figure}[H]
	\centering
	\begin{minted}{c}
void processData(Data* data, int index, int whatToDo) {
	// Add something
	if(whatToDo == 1) {
		/* ... */
	}
	// Delete something
	else if(whatToDo == 2) {
		/* ... */
	}
	// Move something
	else if(whatToDo == 3) {
		/* ... */
	} else {
		/* ... */
	}
}
	\end{minted}
	\caption{Paaas bien. En plus le nom de la fonction nous dit rien sur ce
		qu'elle va faire.}
\end{figure}

\begin{figure}[H]
	\centering
	\begin{minted}{c}
void deleteData(Data* data, int index) {
	/* ... */
}

void addData(Data* data, int index, Data dataToAdd) {
	/* ... */
}

void moveData(Data* data, int oldIndex, int newIndex) {
	/* ... */
}
	\end{minted}
	\caption{Ça fait du bien, même si le terme ``data'' reste trop générique}
\end{figure}

\paragraph{} Si vous voulez frimez auprès de vos potes, vous pouvez appeler
cette règle le ``Principe de responsabilité unique'' ou encore pire la ``loi de
Curly''.

\subsection{Commentaires}

\paragraph{} Une domaine où une bonne partie des gens se foire est le domaine
des commentaires ou de la documentation. La principale fonction des
commentaires n'est pas d'expliquer ce que la partie suivante du code fait. Si
jamais il est nécessaire d'expliquer comment une partie de votre code
fonctionne, c'est probablement qu'il y a un problème dans votre code.

\paragraph{} Comme tout votre code doit se comprendre de par lui-même, la seule
utilité qu'il reste au commentaires est d'expliquer pourquoi cette partie de
code est là, dans quel cas de figure cette fonction va être utilisée, etc\ldots
Écrire des commentaires dans ce style là est très pratique pour les personnes
qui regardent votre code (mais ça nécessite quand même que le reste soit
clair).

\begin{figure}[H]
	\centering
	\begin{minted}{java}
/**
 * A User.
 *
 * This stores the user id, the username and the user's address.
 */
public class UserModel {
	/* ... */
}
	\end{minted}
	\caption{Un exemple de commentaire complètement inutile}
\end{figure}

\paragraph{} La documentation de cette classe est complètement inutile de par
le fait qu'il suffit de regarder quelques lignes en dessous pour voir quels
attributs sont stockés dans la classe.

\begin{figure}[H]
	\centering
	\begin{minted}{java}
/**
 * A User in the database.
 *
 * This class is serialized and deserialized by the Jackson library.
 * This will be used by the Ektorp library to add, update users in the database.
 */
public class UserModel {
	/* ... */
}
	\end{minted}
	\caption{Un exemple de commentaire utile}
\end{figure}
