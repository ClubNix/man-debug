\part{Utiliser \textit{gdb}}%
\label{part:gdb}

\section{Introduction}

\paragraph{} \textit{GDB} est bien plus un outil de la mort qui tue qu'il en a
l'air au premier abord. Entre remonter dans le temps, et être en courant des
modifications de chaque variable du programme, ce ne sont pas les
fonctionnalités qui manquent. Commençons donc sans tarder:

\begin{listing}[H]
	\centering
	\begin{minted}{console}
		user@pc:~ % gdb
		GNU gdb (My Distribution) 8.0
		Copyright (C) 2017 Free Software Foundation, Inc.
		License GPLv3+: GNU GPL version 3 or later <http://gnu.org/licenses/gpl.html>
		This is free software: you are free to change and redistribute it.
		There is NO WARRANTY, to the extent permitted by law.  Type "show copying"
		and "show warranty" for details.
		This GDB was configured as "x86_64-pc-linux-gnu".
		Type "show configuration" for configuration details.
		For bug reporting instructions, please see:
		<https://bugs.gentoo.org/>.
		Find the GDB manual and other documentation resources online at:
		<http://www.gnu.org/software/gdb/documentation/>.
		For help, type "help".
		Type "apropos word" to search for commands related to "word".
		(gdb) _
	\end{minted}
	\caption{Yaay, chuis trop un hacker maintenant}
\end{listing}

\paragraph{} Maintenant que l'on vient de voir ce qu'il se passe lorsqu'on
lance \texttt{gdb}, il serait bien de l'expliquer. GDB marche en ligne de
commande, de la même manière que la ligne de commande sous GNU/Linux ou autre,
mais avec des commandes spécifiques à GDB. Un autre particularité des commandes
de GDB est que l'on est pas obligé de les taper en entier. Ainsi, pour la
commande \texttt{continue} (que l'on verra plus tard), on pourra tout
simplement taper \texttt{c}.

\paragraph{On vient de démarrer gdb, mais on débogue quoi du coup?} Très bonne
question. On peut soit spécifier l'exécutable en executant \texttt{file
	<exécutable>} dans la ligne de commande de gdb, soit en exécutant gdb avec
le nom de l'exécutable: \mintinline{console}{% gdb "exécutable"}.

\section{Le programme d'exemple}

\paragraph{} On va éviter de faire les choses dans le désordre, alors reprenons
depuis le début. On a ce code-ci:

\begin{listing}[H]
	\begin{minted}{c}
		#include <stdio.h>

		int main() {
			char* hello;
			printf("My string: %s\n", hello);
		}
	\end{minted}
	\caption{Peux-tu trouver l'erreur?}
\end{listing}

\paragraph{} Bon, c'est un programme un peu ridicule: il essaye d'afficher une
string qui n'a pas été initialisée. Comme les strings sont des tableaux (donc
des pointeurs) il va essayer de récupérer les caractères stocké à partir de
l'adresse dans la variable \texttt{hello}, qui va très très très probablement
ne pas pointer vers quelque chose qui appartient à notre programme. Du coup:
\textit{Segmentation Fault}.

\paragraph{} Mais supposons que nous ne voyons pas cette erreur. Et supposons
aussi que nous n'avons pas demandé à \textit{GCC} d'activer les warnings avec
\texttt{-Wall} et \texttt{-Wextra}\ldots Comment déboger ça? Tout d'abords, il
faut le compiler avec les symboles de débogage (utilisés par GDB) en activant
l'option \texttt{-g} à GCC:

\begin{listing}[H]
	\begin{minted}[linenos=false]{console}
		user@pc:/tmp % gcc main.c -o my_program -g
	\end{minted}
	\caption{Exemple à ne pas refaire. Il ne faut pas oublier
		\texttt{-Wall} et \texttt{-Wextra}}
\end{listing}

\paragraph{} Un fois que le programme est correctement compilé, on verifie
qu'il plante bien:

\begin{listing}[H]
	\begin{minted}[linenos=false]{console}
		user@pc:/tmp % ./my_program
		[1]    31250 segmentation fault (core dumped)  ./my_program
	\end{minted}
	\caption{Youpi, ça plante bien!}
\end{listing}

\section{Lancer GDB}

\paragraph{} Une fois qu'on a notre programme bien foireux, on peut le
déboguer! Lançons donc ce fameux GDB:

\begin{listing}[H]
	\begin{minted}[linenos=false]{console}
		user@pc:/tmp % gdb ./my_program
		GNU gdb (Gentoo 8.0 vanilla) 8.0
		Copyright (C) 2017 Free Software Foundation, Inc.
		License GPLv3+: GNU GPL version 3 or later <http://gnu.org/licenses/gpl.html>
		This is free software: you are free to change and redistribute it.
		There is NO WARRANTY, to the extent permitted by law.  Type "show copying"
		and "show warranty" for details.
		This GDB was configured as "x86_64-pc-linux-gnu".
		Type "show configuration" for configuration details.
		For bug reporting instructions, please see:
		<https://bugs.gentoo.org/>.
		Find the GDB manual and other documentation resources online at:
		<http://www.gnu.org/software/gdb/documentation/>.
		For help, type "help".
		Type "apropos word" to search for commands related to "word"...
		Reading symbols from ./my_program...done.
		(gdb) _
	\end{minted}
\end{listing}

\paragraph{} Pas beaucoup de choses ont changé depuis la dernière fois\ldots
Mais on voit bien à l'avant-dernière ligne que gdb a bien réussi à lire les
symboles de débogages \texttt{{\textbackslash}o/}. Ces symboles de débogages
permettent à GDB de pouvoir associer le code binaire de l'exécutable au code
source C, C++, etc. Concrètement, il va pouvoir vous dire où vous êtes dans le
programme avec les lignes de codes plutôt que le code assembleur. Plutôt
pratique donc. Il faut donc se souvenir de compiler avec \texttt{-g} quand on
développe le programme.

\paragraph{} GDB a donc bien chargé le programme, du coup étape suivante:
lancer le programme dans GDB. Cela se fait avec la commande \texttt{run} ou
\texttt{r} pour les intimes:

\begin{listing}[H]
	\begin{minted}[linenos=false]{text}
		(gdb) r
		Starting program: /tmp/my_program

		Program received signal SIGSEGV, Segmentation fault.
		0x00007ffff7ab3e86 in strlen () from /lib64/libc.so.6
		(gdb) _
	\end{minted}
	\caption{On revoit la SEGFAULT à nouveau}
\end{listing}

\paragraph{Elles sont où les lignes de codes, tu nous as menti!} Oui bon on ne
les voit pas tout de suite. Mais j'y viens justement. À la prochaine partie.

\section{Naviguer dans les fonctions}

\paragraph{} La première chose que l'on voit de nouveau, sur l'expert
précédent, c'est que GDB nous indique que la fonction responsable de la
SEGFAULT est la fonction \mintinline{c}{strlen} qui nous viens du fichier
\texttt{/lib64/libc.so.6}. Mais on a pas appelé la fonction
\mintinline{c}{strlen}, d'où elle viens alors? Il suffit de demander à GDB
grâce à la commande \texttt{backtrace}, ou \texttt{bt} pour les intimes:

\begin{listing}[H]
	\begin{minted}[linenos=false]{text}
		(gdb) bt
		#0  0x00007ffff7ab3e86 in strlen () from /lib64/libc.so.6
		#1  0x00007ffff7a77bfb in vfprintf () from /lib64/libc.so.6
		#2  0x00007ffff7a7e6be in printf () from /lib64/libc.so.6
		#3  0x00000000004004d5 in main () at main.c:5
		(gdb) _
	\end{minted}
	\caption{On a débloqué la minimap!}
\end{listing}

\paragraph{} On y vois maintenant beaucoup plus clair. On voit que c'est bien
\mintinline{c}{printf} qui fait ça merde et qui appelle ce qu'il veut. On voit
aussi que la fonction \mintinline{c}{printf} a été appelée par notre
\texttt{main} à la ligne 5. On peut checker dans notre éditeur de texte
préféré, et comme prévu, c'est la ligne du \mintinline{c}{printf}. M'enfin
c'est chiant de retourner sur l'éditeur de texte pour voir où on est. Alors on
peut utiliser la commande \texttt{list} de GDB pour le faire à notre place:

\begin{listing}[H]
	\begin{minted}[linenos=false]{text}
		(gdb) list
		1       #include <stdio.h>
		2
		3       int main() {
		4               char* hello;
		5               printf("My string: %s\n", hello);
		6       }
		(gdb) _
	\end{minted}
	\caption{J'ai enfin ouvert les yeux}
\end{listing}

\paragraph{} Bon c'est un peu chiant de faire \texttt{list} à chaque fois, du
coup on va utiliser l'interface cachée de GDB (ne le dites à personnes, c'est
un secret d'État). La commande est \texttt{tui enable}. Après avoir fait cette
commande, on se trouve face à une interface qui ressemble à ça.

\begin{listing}[H]
	\begin{minted}[linenos=false]{text}
		   +---------------------------------------------------------------------+
		   |                                                                     |
		   |                                                                     |
		   |                                                                     |
		   |                                                                     |
		   |                                                                     |
		   |                                                                     |
		   |                                                                     |
		   |                                                                     |
		   |                                                                     |
		   |          [ No Source Available ]                                    |
		   |                                                                     |
		   |                                                                     |
		   |                                                                     |
		   |                                                                     |
		   |                                                                     |
		   |                                                                     |
		   |                                                                     |
		   |                                                                     |
		   +---------------------------------------------------------------------+
		native process 21209 In: strlen                  L??   PC: 0x7ffff7ab3e86
		(gdb) _
	\end{minted}
	\caption{Je vois TOUT!}
\end{listing}

\paragraph{} Note: dans la suite, je ne montrerais pas l'interface dans les
extrais de codes, parce que ça me fais chier.

\paragraph{Je ne vois toujours pas les lignes\ldots} Du calme, du calme. C'est
parce que GDB nous situe toujours dans la fonction \mintinline{c}{strlen}. Il
faut lui dire qu'il faut qu'il nous place dans notre fonction
\mintinline{c}{main}.  Pour cela on peut utiliser les commandes \texttt{up},
\texttt{down} et \texttt{frame} de GDB qui nous font naviguer dans la
\textit{backtrace}:

\begin{listing}[H]
	\begin{minted}[linenos=false]{text}
		(gdb) bt
		#0  0x00007ffff7ab3e86 in strlen () from /lib64/libc.so.6
		#1  0x00007ffff7a77bfb in vfprintf () from /lib64/libc.so.6
		#2  0x00007ffff7a7e6be in printf () from /lib64/libc.so.6
		#3  0x00000000004004d5 in main () at main.c:5
		(gdb) up
		#1  0x00007ffff7a77bfb in vfprintf () from /lib64/libc.so.6
		(gdb) up
		#2  0x00007ffff7a7e6be in printf () from /lib64/libc.so.6
		(gdb) down 2
		#0  0x00007ffff7ab3e86 in strlen () from /lib64/libc.so.6
		(gdb) frame 3
		#3  0x00000000004004d5 in main () at main.c:5
		(gdb) _
	\end{minted}
	\caption{Un peu d'exploration de la \textit{backtrace}}
\end{listing}

\paragraph{} Et maintenant, il nous affiche les lignes! Youpi!

\paragraph{Je n'ai plus mon historique de commande avec les flèches \frownie}
Eh oui, maintenant que l'on a activé l'interface, les flèches sont utilisées
pour bouger la fenêtre qui affiche le code. D'autres touches sont assignées
pour remonter ou descendre dans l'historique: \textit{Ctrl+P} et
\textit{Ctrl+N} (pour Previous et Next).

\paragraph{Mon programme affiche du texte, et ça fout la merde avec l'interface
	de GDB.} Pas de panique, il suffit de taper \textit{Ctrl+L} pour demander à
GDB de réafficher l'interface. Ce n'est pas parfait, mais ça devrait permettre
de vous en sortir dans la plupart des situations.

\section{Savoir continuer}

\todo[inline,color=cyan]{continue}
\todo[inline,color=cyan]{next}
\todo[inline,color=cyan]{step}

\section{Savoir s'arrêter}

\subsection{Les \textit{breakpoints}}

\paragraph{} Les points d'arrêts (ou \textit{breakpoints} en anglais) sont la
base du débogage. Ils sont tellements importants que l'on dispose de plusieures
manières pour pouvoir en placer dans GDB. la première et la plus simple est
d'utiliser la commande \texttt{break} (ou \texttt{b} pour les intimes). Cette
commande prend en paramètre soit un nom de fonction, soit un endroit dans un
fichier qui contient du code source.

\paragraph{} Si on reprend notre exemple précédent, on peut demander à GDB de
s'arrêter à la fonction \mintinline{c}{main} de deux manières:

\begin{listing}
	\begin{minted}[linenos=false]{text}
		(gdb) b main
		Breakpoint 1 at 0x4004bf: file main.c, line 5.
		(gdb) r
		Starting program: /tmp/my_program

		Breakpoint 1, main () at main.c:5
		5               printf("My string: %s", hello);
		(gdb) _
	\end{minted}
	\caption{Je me suis arrêté au \mintinline{c}{main}}
\end{listing}

\begin{listing}
	\begin{minted}[linenos=false]{text}
		(gdb) b main.c:3
		Breakpoint 1 at 0x4004bf: file main.c, line 3.
		(gdb) r
		Starting program: /tmp/my_program

		Breakpoint 1, main () at main.c:5
		5               printf("My string: %s", hello);
		(gdb) _
	\end{minted}
	\caption{Je me suis \textbf{encore} arrêté au \mintinline{c}{main}}
\end{listing}

\paragraph{Je lui ai demandé de m'arrêter au \mintinline{c}{main} mais il m'a
	arrêté deux lignes après} me direz vous. Et vous auriez entièrement raison,
mais ce n'est pas un bug de GDB. Il faut se souvenir qu'une fois compilé, la
représentation du code change, et bien que l'appel à la fonction
\mintinline{c}{printf} se traduise bien en assembleur, ce n'est pas le cas des
déclarations simples, comme la déclaration de la variable \mintinline{c}{hello}
ou même de la fonction \mintinline{c}{main}. Ainsi, GDB va à la prochaine
réelle instruction assembleur ayant une relation avec le code source, qui est
dans ce cas ci l'appel à la fonction \mintinline{c}{printf}.

\todo[inline,color=cyan]{rbreak}
\todo[inline,color=cyan]{info breakpoints}
\todo[inline,color=cyan]{enable / disable breakpoints \ldots}
\todo[inline,color=cyan]{break \ldots if \ldots}
\todo[inline,color=cyan]{condition \ldots \ldots}
\todo[inline,color=cyan]{delete breakpoints \ldots}

% https://sourceware.org/gdb/onlinedocs/gdb/Set-Breaks.html
% https://sourceware.org/gdb/onlinedocs/gdb/Disabling.html
% https://sourceware.org/gdb/onlinedocs/gdb/Conditions.html
% https://sourceware.org/gdb/onlinedocs/gdb/Delete-Breaks.html

\subsection{Les \textit{watchpoints}}

\todo[inline,color=cyan]{watch}
\todo[inline,color=cyan]{info watchpoints}

% https://sourceware.org/gdb/onlinedocs/gdb/Set-Watchpoints.html

\subsection{Les \textit{catchpoints}}

\todo[inline,color=cyan]{catch throw}
\todo[inline,color=cyan]{catch syscall \ldots}
\todo[inline,color=cyan]{catch syscall group:\ldots}

% https://sourceware.org/gdb/onlinedocs/gdb/Set-Catchpoints.html

\subsection{Le \mintinline{c}{printf} du futur}

% https://sourceware.org/gdb/onlinedocs/gdb/Dynamic-Printf.html
