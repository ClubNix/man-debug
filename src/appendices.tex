\part{Annexes}

\section{GDB}

\subsection{À savoir}

\begin{itemize}
	\item Compiler avec l'option \texttt{-g} et de préférence sans
		optimizations (sans \texttt{-O1}, \texttt{-O2}, etc.)
	\item Lancer GDB avec le nom de l'exécutable en paramètre en
		ligne de commande.
	\item Dans l'interface de GDB, utiliser Ctrl+P et Ctrl+N pour
		naviguer dans l'historique.
	\item Dans l'interface de GDB, utiliser Ctrl+L pour réafficher l'écran.
\end{itemize}

\subsection{Cheatsheet}

\begin{table}[H]
	\centering
	\begin{tabular}{lp{9cm}}
		\toprule

		\textbf{Commande} & \textbf{Description}\\

		\midrule

		\texttt{tui enable} & Active l'interface graphique de GDB\\
		\texttt{r(un)} & Lance le programme\\

		\midrule

		\texttt{bt} & Affiche la backtrace\\
		\texttt{up} & Monte d'un niveau dans la backtrace\\
		\texttt{down} & Descend d'un niveau dans la backtrace\\
		\texttt{f(rame)} \textit{i} & Va à la frame \textit{i} dans
			la backtrace\\

		\midrule

		\texttt{c(ontinue)} & Continue l'exécution du programme\\
		\texttt{n(ext)} & Continue l'exécution du programme pas à pas\\
		\texttt{s(tep)} & Continue l'exécution du programme pas à pas en
			rentrant dans les fonctions\\

		\midrule

		\texttt{b(reak)} \textit{func} & Ajoute un point d'arrêt à la
			fonction \textit{func}\\
		\texttt{b(reak)} \textit{fichier}\texttt{:}\textit{ligne} & Ajoute
			un point d'arrêt à un endroit précis dans un fichier
			contenant du code source.\\
		\texttt{b(reak)} \textit{endroit} \texttt{if} \textit{condition} &
			Arrête le programme à l'\textit{endroit} spécifié si
			\textit{condition}\\
		\texttt{wa(tch)} \textit{variable} & Arrête le programme lors du
			changement de valeur de la variable \textit{variable}\\

		\midrule

		\texttt{inf(o) b(reakpoints)} & Affiche la liste des points d'arrêts\\
		\texttt{dis(able) b(reakpoints)} \textit{id} & Désactive le
			point d'arrêt n°\textit{id} (tous si \textit{id} n'est
			pas spécifié)\\
		\texttt{en(able) b(reakpoints)} \textit{id} & Réactive le
			point d'arrêt n°\textit{id} (tous si \textit{id} n'est
			pas spécifié)\\
		\texttt{d(elete) br(eakpoints)} \textit{id} & Supprime le
			point d'arrêt n°\textit{id} (tous si \textit{id} n'est
			pas spécifié)\\

		\midrule

		\texttt{catch throw} & Arrête le programme lors du lancement
			d'une exception C++\\
		\texttt{catch syscall} \textit{nom} & Arrête le programme lors de
			l'appel système \textit{nom}\\
		\texttt{catch syscall group:}\textit{nom} & Arrête le programme lors
			d'un appel système du groupe \textit{nom}\\

		\bottomrule
	\end{tabular}
	\caption{Cheatsheet de GDB}
\end{table}
