\part{Introduction}

\paragraph{}

Oyé oyé aventuriers et aventurières de la programmation en C et C++ qui
souffrent le martyr, causé par une incompréhension entre l'ordinateur et son
utilisateur. Ne craignez plus car ce poly est arrivé. Fini, les
\textit{Segmentation fault}, les comportements impossibles. Tout ça sera du
passé.

\section{Vos nouveaux héros}

% TODO: quick presentation of tools

\section{Types de bugs}

\subsection{Présentation}

\paragraph{}

Mère nature est une salope. Elle nous a conçus de manière imparfaite et cela
nous oblige à tout catégoriser pour que notre pauvre cerveau puisse ne pas
sortir de sa zone de confort. Cependant, dans le cas des bugs, cela nous permet
de savoir quel outil sera le plus probable de nous aider à trouver l'origine du
problème. En voici des plus connus:

\begin{table}[H]
	\centering
	\begin{tabular}{lp{20em}p{9em}}
		\toprule
		\textbf{Nom} & \textbf{Description} & \textbf{Outil(s) à utiliser}\\
		\midrule
		Bohrbug & Avec un nom qui viens du modèle de Bohr pour les atomes,
			cette catégorie contient les bugs ayant un comportement stable,
			apparaissant dans des conditions stables. Ce sont les bugs les plus
			simple à détecter, comprendre et corriger. & \textbf{gdb}\\
		Mandelbug & Venant de la fractale de Mandelbrot, il s'agit d'un bug
			dont les origines sont tellement complexes qu'il paraît avoir un
			comportement chaotique. & \textbf{gdb}, \textbf{valgrind}\\
		Shrödingbug & Un bug qui se manifeste au moment où l'on remarque que
			le programme n'aurait jamais dû marcher. & Un facepalm\\
		Hindenbug & Un bug stupide de la part du développeur avec des
			conséquences catastrophiques. & Beaucoup de facepalm\\
		Heisenbug & Un bug qui change de comportement à partir du moment où
			l'on l'étudie. Ils sont encore plus casse-couilles que la
			description laisse l'entendre. & \textbf{rr}, \textbf{gdb},
			\textbf{valgrind} \\
		\bottomrule
	\end{tabular}%
	\label{tab:bugtypes}
	\caption{Types de bugs et outils appropriés}
\end{table}

\paragraph{}

Notez tout de même qu'il n'y a aucune référence vers des outils de la
partie~\ref{part:before} car ce qui sera présenté dans cette partie sera utile
dans tous les cas de figure.

\subsection{Que faire?}

\subsubsection{Bohrbug}

\paragraph{}

Pour un ``bon vieux bug à l'ancienne'', la meilleure solution est probablement
de réfléchir calmement. Si l'origine du bug ne vous viens pas naturellement, le
mieux est de faire exécuter votre programme étape par étape avant l'apparition
du bug avec \textbf{gdb} (page~\pageref{part:gdb}). Si jamais cela ne suffit
pas, il faut essayer d'observer les changements au cours de l'exécution de
votre programme de certaines variables critiques, cela toujours avec
\textbf{gdb}. Si ce n'est toujours pas suffisant, essayez de voir avec
\textbf{valgrind} (page~\pageref{part:valgrind}) si votre problème ne viens pas
d'une erreur déterministe avec la mémoire, comme par exemple une variable se
fait ``engloutir'' par une mauvaise utilisation d'une variable voisine.

\subsubsection{Mandelbug}

\paragraph{}

La chose la plus importante à faire pour traiter un Mandlebug est de déterminer
les raisons de son apparition. C'est pour ça qu'utiliser \textbf{valgrind}
(page~\pageref{part:valgrind}) est une bonne idée pour vérifier si de la
mémoire non-initialisée a été touchée par exemple. Si cela ne donne rien,
essayer d'exécuter votre programme dans \textbf{gdb} (page~\pageref{part:gdb})
de telle sorte que le bug apparaît, vous pouvez utiliser \textbf{rr}
(page~\pageref{part:rr}) pour cela, et regarder les variables critiques qui
pourraient amener à ce bug.


\subsubsection{Shrödingbug}

\paragraph{}

Les Schrödingbug sont plutôt simple à résoudre puisque l'on vient de voir que
le programme n'aurait jamais dû marcher. Il suffit donc de résoudre le
problème.

\subsubsection{Hindenbug}

\paragraph{}

Les Hindenbug arrivent le plus souvent dans des entreprises où tous les
développeurs ne sont pas forcément intéressés à produire un code propre et
deviennent facilement partisans d'utiliser des méthodes ``douteuses'' afin de
résoudre un problème rapidement (ce qui sera, bien sûr, jamais votre cas). Il
est aussi possible qu'il soit créé par des personnes vouant une haine
particulière envers une personne de votre entreprise ou l'entreprise même. Il
vous faudra alors probablement toucher au code qui n'est pas le vôtre et de
bien comprendre ses subtilités avant de vous lancer dans \textbf{gdb} ou
\textbf{valgrind}, sans oublier de contacter et discuter avec vos collègues.
Mais le plus important, la chose la plus essentielle si ce n'est obligatoire,
est de poster sur un blog vos déboires et la souffrance que vous avez vécu
durant ce qui risque d'être une aventure, et de le partager.

\subsubsection{Heisenbug}

\paragraph{}

Comme vous avez pu le deviner, ces bugs sont une horreur à déboguer et sont
souvent indicateur de plusieurs jours d'arrachage de cheveux. L'outil le plus
efficace qui a été créé pour l'occasion est \textbf{rr}
(page~\pageref{part:rr}). Ce que vous aurez à faire est de faire tourner votre
programme avec \textbf{rr} jusqu'à ce que le bug apparaisse puis utiliser
\textbf{rr} pour pouvoir lancer \textbf{gdb} et essayer de vous en sortir de la
même manière que vous le feriez pour un Bohrbug ou un Mandlebug.

\section{Bonnes pratiques}

% TODO
